\documentclass{anstrans}
%%%%%%%%%%%%%%%%%%%%%%%%%%%%%%%%%%%
\title{Application of the Karhunen-Lo\`{e}ve Transform to the C5G7 benchmark in
the Response Matrix Method}
\author{Richard L. Reed, Jeremy A. Roberts}

\institute{
Mechanical and Nuclear Engineering, Kansas State University, Manhattan, KS
}

\email{rlreed@k-state.edu \and jaroberts@k-state.edu}

% Optional disclaimer: remove this command to hide
%\disclaimer{Notice: This manuscript is a work of fiction. Any resemblance to
%actual articles, living or dead, is purely coincidental.}

%%%% packages and definitions (optional)
\usepackage{graphicx} % allows inclusion of graphics
\usepackage{booktabs} % nice rules (thick lines) for tables
\usepackage{tabulary}
\usepackage{microtype} % improves typography for PDF
\usepackage{xcolor}
\usepackage{amsmath}
\usepackage{bm}
\usepackage{tikz}
\usepackage{verbatim}
\usepackage{braket}
\usepackage[superscript,nospace]{cite}
\usepackage{amssymb}
\usepackage{morefloats}
\usepackage{todonotes}
\usepackage{esvect}
\usepackage[amssymb,cdot]{SIunits}
\usepackage{geometry}

\graphicspath{{figures/}} % Specifies the directory where pictures are stored

% put your own definitions here:
\newcommand{\SN}{S$_N$}
\renewcommand{\vec}[1]{\bm{#1}} %vector is bold italic
\newcommand{\vd}{\bm{\cdot}} % slightly bold vector dot
\newcommand{\grad}{\vec{\nabla}} % gradient
\newcommand{\ud}{\mathop{}\!\mathrm{d}} % upright derivative symbol
%   \newcommand{\cZ}{\cal{Z}}
%   \newtheorem{def}{Definition}[section]
%   ...
\newcommand{\oper}[1]{\mathcal{#1}}
\providecommand{\e}[1]{\ensuremath{\times 10^{#1}}}
\newcommand{\CHAPTER}[1]{Chapter~\ref{#1}}
\newcommand{\EQ}[1]{Eq.~(\ref{#1})}               %-- Eq. (refeq)
\newcommand{\EQUATION}[1]{Equation~(\ref{#1})}    %-- Equation (refeq)
\newcommand{\FIG}[1]{Fig.~\ref{#1}}               %-- Fig. refig
\newcommand{\FIGS}[2]{Fig.~\ref{#1}~and~\ref{#2}} %-- Fig. refig
\newcommand{\FIGURE}[1]{Figure~\ref{#1}}
\newcommand{\TAB}[1]{Table~\ref{#1}}              %-- Table tablref
\newcommand{\EQS}[2]{Eqs.~(\ref{#1})--(\ref{#2})}            %-- Eqs. (refeqs)
\newcommand{\EQUATIONS}[2]{Equations~(\ref{#1})--(\ref{#2})}   %-- Eqs. (refeqs)
\newcommand{\EQSTWO}[2]{Eqs.~(\ref{#1})~and~(\ref{#2})}     %-- Eqs. (refeqs)
\newcommand{\EQUATIONSTWO}[2]{Equations~(\ref{#1})~and~(\ref{#2})}
%-- Eqs. (refeqs
\newcommand{\BOXEQ}[1]{\mbox{\fboxsep=.13in $$
        \framebox{$ #1 $} $$ } }    %-- box around equation
\DeclareMathOperator*{\dotp}{{\scriptscriptstyle \stackrel{\bullet}{{}}}}

\begin{document}
\section{Introduction}

Previous work used the Karhunen-Lo\`{e}ve Transform (KLT) to create
basis sets for representing the energy variable in the eigenvalue response
matrix method (ERMM) and demonstrated the efficiency of such basis sets for
1-D problems \cite{annualANS, reedThesis}.
 The present work provides an extension of the KLT basis sets to the 2-D, C5G7
benchmark problem. ERMM solves the reactor eigenvalue equation,
\begin{equation}
  \mathcal{T} \phi(\bm{\rho}) =
    \frac{1}{k} \mathcal{F} \phi(\bm{\rho}) \, ,
  \label{eq:global}
\end{equation}
where $\bm{\rho}$ contains the relevant phase space (i.e., space, angle, and
energy),
by decomposing the domain into independent nodes linked through approximate
boundary conditions that are based on truncated, orthogonal basis expansions.
The
success of ERMM in realistic applications is highly dependent on the
selection of orthogonal bases in each phase space variable.  Basis
sets that capture
high-fidelity transport solutions with low-order expansions are ideal.
In what follows, ERMM and KLT basis are briefly reviewed, and their application
to a modified C5G7 problem is presented.

\section{Methods}

\subsection{Overview of ERMM}

Suppose the global problem of \EQ{eq:global} is defined over a
volume, $V$, which can be decomposed into $N$ disjoint, nodal subvolumes,
$V_i$, that satisfy $V = V_1 \bigcup V_2 \bigcup \ldots \bigcup V_N$.
Then a local transport problem for the $i$th node is defined
\begin{equation}
  \mathcal{T} \phi(\bm{\rho}_i) =
    \frac{1}{k} \mathcal{F} \phi(\bm{\rho}_i) \, ,
  \label{eq:local}
\end{equation}
subject to the incident current boundary condition
\begin{equation}
  J^{\mathrm{}}_{-} (\bm{\rho}_{is}) =
    J^{\mathrm{global}}_{-}(\bm{\rho}_{is}) \, ,
  \label{eq:localbc}
\end{equation}
where $J_{-} (\bm{\rho}_{is}) $ is the incident angular current on
surface $s$.  To reduce the size of the problem space, the local boundary
currents are represented as a truncated
expansion in an
orthogonal basis, $P_m(\bm{\rho}_{is}), \,\,\, m = 0, \, 1, \, \ldots$,
defined over the surface phase space $\bm{\rho}_{is}$.
% is used subject to
% \begin{equation}
%    \int d\rho_{is} P_m(\bm{\rho}_{is})P_n(\bm{\rho}_{is}) = \delta_{mn} \, .
% \end{equation}
By solving Eq. \ref{eq:local} subject to the $m$th order
incident condition
\begin{equation}
 J^{}_{-} (\bm{\rho}_{is}) = P_m(\bm{\rho}_{is}) \,
\end{equation}
on one surface, $s$, and vacuum elsewhere,
a response function is defined as
\begin{equation}
       r^{ms}_{im's'} = \int d\rho_{is'} P_{m'}(\bm{\rho}_{is'})
        J_{+} (\bm{\rho}_{is'}) \, ,
\label{eq:responsefunction}
\end{equation}
where $J^+$ is the outgoing angular current.  The
response function $r^{ms}_{im's'}$
is the $m'$th order current response out of
surface $s'$ due to an $m$th order incident current on
surface $s$.

By expressing the incident and outgoing currents, $J_{\mp}$, as truncated
expansions in the basis, global neutron balance can be represented by the
response matrix equation
\begin{equation}
  \mathbf{M}\mathbf{R}(k)\mathbf{J_-}  = \lambda \mathbf{J_-} \, ,
\label{eq:erme}
\end{equation}
where $\mathbf{R}$ contains response functions, $\mathbf{J}_{-}$ contains the
incident angular current expansion coefficients, $\mathbf{M}$ is a matrix that
redirects outgoing currents as incident currents of neighbors, and $\lambda$ is
an eigenvalue that approaches unity as $k$ approaches its correct value. The
$k$-eigenvalue is computed iteratively by alternately solving for
$\mathbf{J}_{-}$ and updating $k$ based on balance \cite{RobertsSerment}.

\subsection{Expanding in Energy}

The present work focused solely on expansion of the energy variable.
Traditionally, RMMs have used a full multigroup
representation of the energy variable, which is to say no truncation for
the energy basis is used.  Response matrix methods have been used
successfully with a variety of energy group structures, ranging from three to
190 groups \cite{ishii2009tdd, forget2006tdh, forget2004hcm}.

In general, RRM is prohibitively expensive unless the problem space is
greatly reduced by truncating the boundary conditions to each node.  Thus, an
effective basis set will permit the use of a low-order expansion without
introducing a large error.  Realistic problems require on the order of dozens
of energy groups
or more, and a basis that can capture many-group fidelity with
many fewer energy degrees of freedom would be of substantial
value to RMM.

Recent work investigated the use of discrete Legendre polynomials (DLP) and
modified DLP (mDLP) for expansion in energy \cite{Roberts2014}.  The mDLP basis
modifies the DLP basis by superimposing a ``shape'' vector on each basis vector.
 The shape vector is chosen to be representative of the vector subject to
expansion, i.e., the energy-dependent angular current.  Previously, a spatially
averaged energy spectrum from a representative lattice problem has been used
with moderate success \cite{Roberts2014}.

\subsection{The  Karhunen-Lo\`{e}ve Energy Basis}

A more powerful approach (compared to mDLP) of incorporating physics into the
basis functions is to use the Karhunen-Lo\`{e}ve transform (KLT)
\cite{Dony2001}.  The central goal of KLT is to approximate a discrete or
continuous function $f(x)$ as a truncated expansion in an orthogonal basis whose
functions yield the best possible $n$th-order approximation in terms of
least-squares error for all values of $n$ by extracting information about the
representative spectra in the problem space.  In some applications, such as
image compression, the function $f$ is predetermined, e.g., a set of pixel
values, while in other applications, such as reduced-order modeling, the
function $f$ is not known.  While details of its use in these applications may
differ, KLT is fundamentally related to the singular value decomposition (SVD)
\cite{reedThesis}.

The implementation algorithm for computing the basis
functions, $P^i_{\text{KLT}}(:)$, begins by forming the data matrix
$\mathbf{D}$.  Let the $n$th current vector be denoted by
$\mathbf{d}_n$.  These
vectors, referred to as snapshots, are combined to form the
matrix $\mathbf{D} \in \mathbb{R}^{G\times N}$, which is defined as $\mathbf{D}
= [\mathbf{d}_1,\, \mathbf{d}_2,\, \ldots, \, \mathbf{d}_N]$
\cite{Meyer2002}, where $G$ is the number of energy groups and $N$ is the total
number of snapshots. Then the matrix
$\mathbf{B} \in \mathbb{R}^{N\times N}$ is defined as
\begin{equation}
    \mathbf{B} = \mathbf{D}^{\intercal}\mathbf{D} \, .
    \label{eq:KLT}
\end{equation}
Next, find the eigenvectors corresponding to the largest eigenvalues of the
matrix
$\mathbf{B}$,
\begin{equation}
    \label{eq:SVD_of_B}
    \mathbf{B} = \mathbf{Q}\mathbf{\Lambda}\mathbf{Q}^{-1}\, ,
\end{equation}
where $\mathbf{Q}$ from \EQ{eq:SVD_of_B} is equal
to $\mathbf{V}$
from the SVD of $\mathbf{D}$, and is thus the set of
eigenvectors of
$\mathbf{D}$. The eigenvectors $\mathbf{q}_j$ (columns of $\mathbf{Q}$ from
\EQ{eq:SVD_of_B}), are
then multiplied by the data matrix to form the basis vectors $P^j_{KLT}(:)$,
i.e.,
\begin{equation}
    P^j_{KLT}(:) = \mathbf{D}\mathbf{q}_j \, ,
\end{equation}
which are subsequently orthonormalized.  Then, an approximate representation of
an arbitrary $N$-vector $\mathbf{f}$ in the basis is
\begin{equation}
    \mathbf{f} \approx \sum_j a_j P^j_{KLT}(:) \, \quad \text{where} \quad a_j
    =
    \mathbf{f}^T P^j_{KLT}(:) \, .
    \label{eq:KLT_def}
\end{equation}

Since KLT requires information about the solution {\it a priori}, small models
that are similar to the test problem are developed to provide the necessary
information.  This work identifies how similar the small models need be to
provide an adequate representation of the test problem. KLT snapshots are
energy group-dependent flux or current vectors from several spatial cells within
one or more representative models.

\subsection{Test Problem}

To test the application of KLT in 2-D space, the C5G7 benchmark problem was
adapted for use.  This benchmark consists of modeling a
quarter core that has four, $17\times17$-pin
assemblies \cite{C5G7}.  The configuration is detailed in
\FIG{fig:C5G7_config}.  Each of the blocks in \FIG{fig:C5G7_config}
represent either a fuel assembly or an equivalent area of
moderator.  Each pincell is broken up into a $7\times7$
Cartesian mesh. Thus, each pincell contained 49 spatial cells and provided 49
energy dependent snapshots of each type (scalar flux, partial current, etc.).
The cladding for the pincells was
homogenized into the fuel part of the pincell.

The standard C5G7 benchmark uses seven-group cross-sections, which precludes
the need for an energy approximation.  Thus, a cross section library was
generated with SCALE 5.1 in the SCALE 44-group format \cite{Scale}.  All other
aspects of the benchmark were unchanged.

\begin{figure}[htb]
    \centering
    \begin{tikzpicture}[scale=0.1, every node/.style={scale=1}]
        \filldraw[xshift=42.84 cm, yshift=0 cm, fill=blue!30!white, draw=black]
        (0, 0) rectangle (21.42,21.42) node[pos=.5] {Mod};
        \filldraw[xshift=42.84 cm, yshift=21.42 cm, fill=blue!30!white,
draw=black]
        (0, 0) rectangle (21.42,21.42) node[pos=.5] {Mod};
        \filldraw[xshift=42.84 cm, yshift=42.84 cm, fill=blue!30!white,
draw=black]
        (0, 0) rectangle (21.42,21.42) node[pos=.5] {Mod};
        \filldraw[xshift=21.42 cm, yshift=42.84 cm, fill=blue!30!white,
draw=black]
        (0, 0) rectangle (21.42,21.42) node[pos=.5] {Mod};
        \filldraw[xshift=0 cm, yshift=42.84 cm, fill=blue!30!white, draw=black]
        (0, 0) rectangle (21.42,21.42) node[pos=.5] {Mod};



        \foreach \x in {0,1.26,...,20.16}
        \foreach \y in {0,1.26,...,20.16}
        \filldraw[xshift=\x cm, yshift=\y cm, fill=blue!30!white,
        draw=black] (0, 0) rectangle (1.26,1.26) node[pos=.5] {};
        \foreach \x in {0,1.26,...,20.16}
        \foreach \y in {0,1.26,...,20.16}
        \filldraw[xshift=\x cm, yshift=\y cm, fill=green!30!white,
        draw=black] (.63,.63) circle (.5) node[pos=.5] {};
        \foreach \x in {5*1.26, 8*1.26, 11*1.26}
        \foreach \y in {2*1.26, 14*1.26}
        \filldraw[xshift=\x cm, yshift=\y cm, fill=blue!30!white,
        draw=black] (.63,.63) circle (.5) node[pos=.5] {};
        \foreach \x in {3*1.26, 13*1.26}
        \foreach \y in {3*1.26, 13*1.26}
        \filldraw[xshift=\x cm, yshift=\y cm, fill=blue!30!white,
        draw=black] (.63,.63) circle (.5) node[pos=.5] {};
        \foreach \x in {2*1.26, 5*1.26, 8*1.26, 11*1.26, 14*1.26}
        \foreach \y in {5*1.26, 8*1.26, 11*1.26}
        \filldraw[xshift=\x cm, yshift=\y cm, fill=blue!30!white,
        draw=black] (.63,.63) circle (.5) node[pos=.5] {};



        \foreach \x in {21.42,22.68,...,41.58}
        \foreach \y in {21.42,22.68,...,41.58}
        \filldraw[xshift=\x cm, yshift=\y cm, fill=blue!30!white,
        draw=black] (0, 0) rectangle (1.26,1.26) node[pos=.5] {};
        \foreach \x in {21.42,22.68,...,41.58}
        \foreach \y in {21.42,22.68,...,41.58}
        \filldraw[xshift=\x cm, yshift=\y cm, fill=green!30!white,
        draw=black] (.63,.63) circle (.5) node[pos=.5] {};
        \foreach \x in {22*1.26, 25*1.26, 28*1.26}
        \foreach \y in {19*1.27, 31*1.26}
        \filldraw[xshift=\x cm, yshift=\y cm, fill=blue!30!white,
        draw=black] (.63,.63) circle (.5) node[pos=.5] {};
        \foreach \x in {20*1.26, 30*1.26}
        \foreach \y in {20*1.26, 30*1.26}
        \filldraw[xshift=\x cm, yshift=\y cm, fill=blue!30!white,
        draw=black] (.63,.63) circle (.5) node[pos=.5] {};
        \foreach \x in {19*1.26, 22*1.26, 25*1.26, 28*1.26, 31*1.26}
        \foreach \y in {22*1.26, 25*1.26, 28*1.26}
        \filldraw[xshift=\x cm, yshift=\y cm, fill=blue!30!white,
        draw=black] (.63,.63) circle (.5) node[pos=.5] {};



        \foreach \x in {0,1.26,...,20.16}
        \foreach \y in {21.42,22.68,...,41.58}
        \filldraw[xshift=\x cm, yshift=\y cm, fill=blue!30!white,
        draw=black] (0, 0) rectangle (1.26,1.26) node[pos=.5] {};
        \foreach \x in {0,1.26,...,20.16}
        \foreach \y in {21.42,22.68,...,41.58}
        \filldraw[xshift=\x cm, yshift=\y cm, fill=red!30!white, draw=black]
        (.63,.63) circle (.5) node[pos=.5] {};
        \foreach \x in {1.26,2.52,...,20.16}
        \foreach \y in {22.68,23.94,...,41.58}
        \filldraw[xshift=\x cm, yshift=\y cm, fill=red!60!white, draw=black]
        (.63,.63) circle (.5) node[pos=.5] {};
        \foreach \x in {5*1.26,6*1.26,7*1.26,8*1.26,9*1.26,10*1.26,11*1.26}
        \foreach \y in {20*1.26,30*1.26}
        \filldraw[xshift=\x cm, yshift=\y cm, fill=red!90!white, draw=black]
        (.63,.63) circle (.5) node[pos=.5] {};
        \foreach \x in
        {4*1.26,5*1.26,6*1.26,7*1.26,8*1.26,9*1.26,10*1.26,11*1.26,12*1.26}
        \foreach \y in {21*1.26,29*1.26}
        \filldraw[xshift=\x cm, yshift=\y cm, fill=red!90!white, draw=black]
        (.63,.63) circle (.5) node[pos=.5] {};
        \foreach \x in {3.78,5.04,...,16.38}
        \foreach \y in {27.72,28.98,...,36.46}
        \filldraw[xshift=\x cm, yshift=\y cm, fill=red!90!white, draw=black]
        (.63,.63) circle (.5) node[pos=.5] {};
        \foreach \x in {5*1.26, 8*1.26, 11*1.26}
        \foreach \y in {19*1.26, 31*1.26}
        \filldraw[xshift=\x cm, yshift=\y cm, fill=blue!30!white,
        draw=black] (.63,.63) circle (.5) node[pos=.5] {};
        \foreach \x in {3*1.26, 13*1.26}
        \foreach \y in {20*1.26, 30*1.26}
        \filldraw[xshift=\x cm, yshift=\y cm, fill=blue!30!white,
        draw=black] (.63,.63) circle (.5) node[pos=.5] {};
        \foreach \x in {2*1.26, 5*1.26, 8*1.26, 11*1.26, 14*1.26}
        \foreach \y in {22*1.26, 25*1.26, 28*1.26}
        \filldraw[xshift=\x cm, yshift=\y cm, fill=blue!30!white,
        draw=black] (.63,.63) circle (.5) node[pos=.5] {};



         \foreach \x in {21.42,22.68,...,41.58}
        \foreach \y in {0,1.26,...,20.16}
        \filldraw[xshift=\x cm, yshift=\y cm, fill=blue!30!white,
        draw=black] (0, 0) rectangle (1.26,1.26) node[pos=.5] {};
        \foreach \x in {21.42,22.68,...,41.58}
        \foreach \y in {0,1.26,...,20.16}
        \filldraw[xshift=\x cm, yshift=\y cm, fill=red!30!white, draw=black]
        (.63,.63) circle (.5) node[pos=.5] {};
        \foreach \x in {22.68,23.94,...,41.58}
        \foreach \y in {1.26,2.52,...,20.16}
        \filldraw[xshift=\x cm, yshift=\y cm, fill=red!60!white, draw=black]
        (.63,.63) circle (.5) node[pos=.5] {};
        \foreach \x in {22*1.26,23*1.26,24*1.26,25*1.26,26*1.26,27*1.26,28*1.26}
        \foreach \y in {3*1.26,13*1.26}
        \filldraw[xshift=\x cm, yshift=\y cm, fill=red!90!white, draw=black]
        (.63,.63) circle (.5) node[pos=.5] {};
        \foreach \x in
{21*1.26,22*1.26,23*1.26,24*1.26,25*1.26,26*1.26,27*1.26,28*1.26,29*1.26}
        \foreach \y in {4*1.26,12*1.26}
        \filldraw[xshift=\x cm, yshift=\y cm, fill=red!90!white, draw=black]
        (.63,.63) circle (.5) node[pos=.5] {};
        \foreach \x in {25.2,26.46,...,39.06}
        \foreach \y in {6.3,7.56,...,15.04}
        \filldraw[xshift=\x cm, yshift=\y cm, fill=red!90!white, draw=black]
        (.63,.63) circle (.5) node[pos=.5] {};
        \foreach \x in {22*1.26, 25*1.26, 28*1.26}
        \foreach \y in {2*1.26, 14*1.26}
        \filldraw[xshift=\x cm, yshift=\y cm, fill=blue!30!white,
        draw=black] (.63,.63) circle (.5) node[pos=.5] {};
        \foreach \x in {20*1.26, 30*1.26}
        \foreach \y in {3*1.26, 13*1.26}
        \filldraw[xshift=\x cm, yshift=\y cm, fill=blue!30!white,
        draw=black] (.63,.63) circle (.5) node[pos=.5] {};
        \foreach \x in {19*1.26, 22*1.26, 25*1.26, 28*1.26, 31*1.26}
        \foreach \y in {5*1.26, 8*1.26, 11*1.26}
        \filldraw[xshift=\x cm, yshift=\y cm, fill=blue!30!white,
        draw=black] (.63,.63) circle (.5) node[pos=.5] {};

        \draw[xshift=3*21.42cm,yshift=1.5*21.42cm] node[right]
        {\rotatebox{-90}{Vacuum}};
        \draw[yshift=1.5*21.42cm] node[left] {\rotatebox{90}{Reflect}};
        \draw[xshift=1.5*21.42cm,yshift=3*21.42cm] node[center, above]
{{Vacuum}};
        \draw[xshift=1.5*21.42cm] node[center, below] {{Reflect}};
    \end{tikzpicture}
    \caption{Configuration for C5G7 benchmark.  Each square represents the area
of a
             $17\times17$ pin assembly}
    \label{fig:C5G7_config}
\end{figure}

For this test problem, the spatial and angular orders for the SERMENT
\cite{RobertsSerment} reference solution and the RMM solutions for each
snapshot case were set to fourth order. It is assumed that the relative
error of the snapshot cases is a function of the energy expansion alone since
all other aspects of the problem are identical between the reference solution
and each test case.  It is impossible to be sure if this assumption is valid
without computing the problem at better resolution.

\subsection{Generation of Snapshots}

\begin{table*}[htb]
    \centering
    \caption{Summary of snapshot models for C5G7 test Problem}
    \begin{tabulary}{\linewidth}{c | L}\toprule
        Abbreviation         & Model to generate snapshots \\ \midrule
        Reduced Full-Core    & Spatially-averaged snapshots from whole core
model (i.e.,         the test problem) \\
        Combined-Assemblies  & Snapshots from assemblies used in core
                               configuration \\
        Combined-Pins        & Snapshots from pins used in core
                               configuration combined with the pin
                               junctions\\
        Small-Core           & Snapshots from the small core model \\
        Small-Assemblies     & Snapshots from the small assemblies used in the
                               small core configuration \\
        Reduced Small-Core   & Spatially-averaged snapshots from the small core
model \\
        1-D Approximation    & Snapshots from the 1-D approximation to the C5G7
                               benchmark \\
        \bottomrule
    \end{tabulary}
    \label{tab:C5G7snapshots}
\end{table*}

There are several models that are readily apparent choices to simplify the C5G7
benchmark for use in generating snapshots.  The snapshots models for
this test problem are summarized in \TAB{tab:C5G7snapshots}.  The first model
was to use snapshots from the test problem itself to gain an understanding of
the best possible snapshots.  However, the number of snapshots obtained for this
model is prohibitively large, even after removing all duplicate snapshots by
model symmetry, and thus unusable in the raw state.  However, by spatially
averaging the snapshots over a pincell, the number of
snapshots is reduced by a factor of 49, and thus becomes a manageable set with
which to create basis functions.

The Combined-Assemblies model was formed by taking snapshots from each assembly
and combining the snapshots together.  For this snapshot model, each assembly
was modeled with reflective conditions on all sides.  The next model was the
Combined-Pins model, which derives snapshots from individual pincells for each
fuel type.  The pincell snapshots are combined with snapshots from junctions of
pincells. These junctions are modeled as
$2\times2$-pin assemblies with
reflective conditions.  There were a total of five different materials, thus
ten unique junctions to model.

The next model was to create a small version of the C5G7 core, which
has the same assembly configuration shown in \FIG{fig:C5G7_config}, but
instead uses $8\times8$-pin assemblies \cite{reedThesis}.  An additional model
arises from the small core model, which is to combine snapshots from each of the
small assemblies.  This is not expected to perform as well as the full-sized,
combined-assembly model, but it will be quicker to create the snapshots, and
thus the basis functions.  A third model from the small core was to spatially
average the snapshots over each pincell similarly to the spatially-reduced,
full-core solution.  The reduced, small-core model was used to approximate the
non-spatially averaged, full-core model by comparing the reduced and non-reduced
small-core models.

The final snapshot model for the C5G7 test problem is to create a
similar 1-D model \cite{reedThesis} with three sections: UO$_2$, MOX, and
moderator.  The model is comprised of 51 pins that are each of width 1.26 cm.
This problem was created with a reflective boundary condition on the UO$_2$
side of the model and a vacuum boundary condition on the moderator side of the
model.

In general, snapshot models should
be computationally quick to solve, yet similar to the test problem.  Timing
studies provide little insight in this case because the snapshots are orders of
magnitude quicker to generate than the response matrices. Furthermore, it is
impossible to know {\it a priori} how many basis functions are required for
accuracy before generating the response matrices.

\section{Results and Analysis}

The goal of this work was to see how applicable the method of snapshots and
the KLT was to 2-D problems.  In the interest of time, the results for
this section have been truncated to 10th order.  The overarching goal of the
work was to achieve sub-$0.1\%$ relative pin
power errors while reducing the necessary energy degrees of freedom by an order
of magnitude.  As such, 10th order is sufficient to determine if KLT can reach
the goal in a 2-D problem-space.  As mentioned previously, we assume any
changes in the solution observed when using
various energy bases is assumed to be a function of the energy basis alone.

\begin{figure}[tb]
    \centering
    \includegraphics[trim=.1cm .25cm 2.0cm .4cm clip=true,
    totalheight=0.28\textheight]{Figures/c/c5g7/rf_plots/phi/%
        energy_basis_comparison_fission-10}
    \caption{Relative error in fission density for 44-group, C5G7 test problem
using snapshot of only $\phi$.}
    \label{fig:c5g7-flux-only}
\end{figure}

% \begin{figure}[tb]
%     \centering
%     \includegraphics[trim=.1cm .25cm 2.0cm .4cm clip=true,
%     totalheight=0.28\textheight]{Figures/s/c5g7/rf_plots/partial/%
%         partial_energy_basis_comparison_fission-10}
%     \caption{Relative error in fission density for 44-group, C5G7 test
%         using snapshot of $J_{\text{up}}$ and $J_{\text{down}}$.}
%     \label{fig:c5g7-current}
% \end{figure}

\begin{figure}[tb]
    \centering
    \includegraphics[trim=.1cm .25cm 2.0cm .4cm clip=true,
    totalheight=0.28\textheight]{Figures/c/c5g7/rf_plots/partial/%
        partial_energy_basis_comparison_fission-10}
    \caption{Relative error in fission density for 44-group, C5G7 test problem
        using snapshot of $\phi$, $J_{\text{up}}$, and $J_{\text{down}}$.}
    \label{fig:c5g7-combined}
\end{figure}

\begin{figure}[tb]
    \centering
    \includegraphics[trim=.1cm .25cm 2.0cm .4cm clip=true,
    totalheight=0.28\textheight]{Figures/c/c5g7/rf_plots/phi/%
        energy_basis_comparison_pinpower-10}
    \caption{Relative error in pin power for 44-group, C5G7 test problem
using snapshot of only $\phi$.}
    \label{fig:c5g7-flux-only-pp}
\end{figure}

% \begin{figure}[tb]
%     \centering
%     \includegraphics[trim=.1cm .25cm 2.0cm .4cm clip=true,
%     totalheight=0.28\textheight]{Figures/s/c5g7/rf_plots/partial/%
%         partial_energy_basis_comparison_pinpower-10}
%     \caption{Relative error in pin power for 44-group, C5G7 test problem
% using snapshot of both $J_{\text{up}}$ and $J_{\text{down}}$.}
%     \label{fig:c5g7-partial-pp}
% \end{figure}

\begin{figure}[tb]
    \centering
    \includegraphics[trim=.1cm .25cm 2.0cm .4cm clip=true,
    totalheight=0.28\textheight]{Figures/c/c5g7/rf_plots/partial/%
        partial_energy_basis_comparison_pinpower-10}
    \caption{Relative error in pin power for 44-group, C5G7 test problem
using snapshot of both $\phi$, $J_{\text{up}}$, and $J_{\text{down}}$.}
    \label{fig:c5g7-combined-pp}
\end{figure}

Each curve shown in this section except for `DLP' and `mDLP' was
generated using the KLT basis with distinct snapshot data. The mDLP results
represent the best case previously observed \cite{Roberts2014}.
\FIGURE{fig:c5g7-flux-only} presents the results from using snapshots of only
the scalar flux $\phi$ in the KLT basis generation in order to approximate the
nodal fission densities.  Many of the snapshot models fall short of the goal of
sub 0.1\% relative error in the first 10 degrees of freedom, but some snapshot
models perform encouragingly well. The Small-Core performs the best at
9th order, but the Combined-Pins model performs encouragingly well despite its
simplicity.  Further, the Small-Core models performs similarly to the Reduced
Small-Core, thus we may assume that the Reduced Full-Core model provides a
close approximation to the Full-Core results.

In general, the results are improved by including snapshots of the partial
current $J$ for basis generation as shown in \FIG{fig:c5g7-combined}.  The
Combined-Pins model performs quite well comparatively, and achieves the goal by
9th order.  However, it is unknown if the relative error will remain below the
goal at higher orders without running additional calculations.  Further, the
Small-Core results are omitted because the number of snapshots was too high to
apply KLT without averaging, thus we must take the Reduced Small-Core model as
an approximation for the performance of the Small-Core model.

Many of the same trends are observed for pin-power errors as shown in
\FIGS{fig:c5g7-flux-only-pp}{fig:c5g7-combined-pp}.  The
former uses snapshots of only $\phi$, while the latter also includes
snapshots of $J$.  Since pin powers are a local quantity, it
was expected (and observed) that the errors are larger than for the more global
nodal fission densities.

\section{Conclusion}

For 2-D problems, many KLT basis outperform the mDLP and DLP bases for ERMM
energy expansions because the KLT maximizes the information contained in
low-order expansions.  In general, the results are
improved by including the partial current in generation of snapshots as
opposed to using only the scalar flux snapshots.

An effective snapshot model is one that can capture the relevant physics of
the problem while being computationally cheap.  The Combined-Pins
model performed quite well despite its simplicity likely because information
about each fuel type as well as the junction between materials were used in
the construction.

It appears that for heterogeneous problems such as the C5G7 benchmark, a basis
expansion in the energy variable requires more degrees of freedom to reach the
goal of sub-0.1\% relative error as compared to the 1D problems previously
explored \cite{annualANS, reedThesis}.  Future work in this area would entail
expanding this application to additional group structures to ensure the results
are not group dependent.  Additionally, it would be interesting to explore
applications of the KLT to expansion in other phase space variables.

\section{Acknowledgements}
The work of the first author was supported by the Kansas State University
Nuclear Research Fellowship Program, generously sponsored by the U.S. Nuclear
Regulatory Commission (Grant NRC-HQ-84-14-G-0033).

%\section*{References}

\bibliographystyle{ans}
\bibliography{bibliography}
\end{document}